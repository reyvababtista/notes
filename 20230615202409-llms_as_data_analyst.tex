% Created 2023-06-21 Wed 16:42
% Intended LaTeX compiler: pdflatex
\documentclass[11pt]{article}
\usepackage[utf8]{inputenc}
\usepackage[T1]{fontenc}
\usepackage{graphicx}
\usepackage{longtable}
\usepackage{wrapfig}
\usepackage{rotating}
\usepackage[normalem]{ulem}
\usepackage{amsmath}
\usepackage{amssymb}
\usepackage{capt-of}
\usepackage{hyperref}
\usepackage{minted}
\author{Reyva Babtista}
\date{\today}
\title{LLMs as data analyst}
\hypersetup{
 pdfauthor={Reyva Babtista},
 pdftitle={LLMs as data analyst},
 pdfkeywords={},
 pdfsubject={},
 pdfcreator={Emacs 28.2 (Org mode 9.6.1)}, 
 pdflang={English}}
\begin{document}

\maketitle
\tableofcontents


\section{Cheng, L., Li, X., \& Bing, L. (2023). Is GPT-4 a Good Data Analyst? \cite{chengGPT4GoodData2023}}
\label{sec:org7dcba34}
This papaer talked about comparing GPT-4's
performance of data analysis to actual data analyst. Their GPT-4 as data
analyst's experiment framework:
\begin{enumerate}
\item code generation (for data collection and data visualization)
\item code execution
\item analysis generation (for data analysis)
\end{enumerate}
They used nvBench dataset to evaluate their experiments. Their evaluation
metrics include figure evaluation (information correctness, chart type
correctness, aesthetics) and analysis evaluation (correctness, alignment,
complexity, fluency).

Although they found that GPT-4 perfomances are comparable to human data
analysts, GPT-4 still suffer from hallucination or poor calculation accuracy.
\end{document}